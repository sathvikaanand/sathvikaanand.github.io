%%%%%%%%%%%%%%%%%%%%%%%%%%%%%%%%%%%%%%%%%
% Medium Length Professional CV
% LaTeX Template
% Version 3.0 (December 17, 2022)
%
% This template originates from:
% https://www.LaTeXTemplates.com
%
% Author:
% Vel (vel@latextemplates.com)
%
% Original author:
% Trey Hunner (http://www.treyhunner.com/)
%
% License:
% CC BY-NC-SA 4.0 (https://creativecommons.org/licenses/by-nc-sa/4.0/)
%
%%%%%%%%%%%%%%%%%%%%%%%%%%%%%%%%%%%%%%%%%

%----------------------------------------------------------------------------------------
%	PACKAGES AND OTHER DOCUMENT CONFIGURATIONS
%----------------------------------------------------------------------------------------

\documentclass[
	%a4paper, % Uncomment for A4 paper size (default is US letter)
	11pt, % Default font size, can use 10pt, 11pt or 12pt
]{resume} % Use the resume class

\usepackage{tgtermes} % Use the EB Garamond font
\usepackage{hyperref}

\usepackage[symbol]{footmisc}


%------------------------------------------------

\name{Sathvika Anand} % Your name to appear at the top

% You can use the \address command up to 3 times for 3 different addresses or pieces of contact information
% Any new lines (\\) you use in the \address commands will be converted to symbols, so each address will appear as a single line.

\address{1076 Alderbrook Lane \\ San Jose, California, 95129} % Main address
\address{(408)~$\cdot$~802~$\cdot$~4292 \\ sanand@g.hmc.edu} % Contact information
%----------------------------------------------------------------------------------------
\begin{document}
%----------------------------------------------------------------------------------------
%	EDUCATION SECTION
%----------------------------------------------------------------------------------------

\begin{rSection}{Education}
	
	\textbf{Harvey Mudd College } \hfill August 2020-Present \\ 
	B.S. in Computer Science expected May 2024\\
	Concentration in Linguistics \\
        Overall GPA: 3.758
 \smallskip \\
\textit{Relevant coursework:} Algorithms, Natural Language Processing, Machine Learning with Neural Signals, Phonetics, Phonology, Writing Systems, Data Structures and Program Development, Computability and Logic, Computer Systems, Discrete Mathematics, Linear Algebra, Statistics and Probability, Programming Languages, User Centered Design \& Research, System Security

	
\end{rSection}

\begin{rSection}{Publications}

    Nile Phillips$^{*}$, \textbf{Sathvika Anand$^{*}$}, Eloise Burtis, Manisha Goel, Michelle Zemel, Alexandra Schofield. `Cheap Talk: Topic Analysis of CSR Themes on Corporate Twitter'. \textsl{Accepted at ECONLP at LREC-COLING 2024}
    
    \textit{* indicates co-lead authorship}
\end{rSection}
%----------------------------------------------------------------------------------------
%	WORK EXPERIENCE SECTION
%----------------------------------------------------------------------------------------

\begin{rSection}{Research Experience}

	\begin{rSubsection}{EconText Lab}{August 2023 - Present}{Research Assistant, Harvey Mudd College \& Pomona College}{Claremont, CA}
            \item Worked under Professor Alexandra Schofield in the EconText lab, in collaboration with Pomona College
		\item Utilized computational techniques, including sentiment analysis, vagueness scores, and regression models, to analyze a substantial Twitter data corpus comprising of around one million tweets from corporate accounts. 
		\item Uncovered relationships between Corporate Social Responsibility (CSR) communication on Twitter and actual  Environmental, Social, and Governance (ESG) efforts, leveraging insights from the data analysis. 
		\item This work is in poster form, with a paper in progress.
	\end{rSubsection}

%------------------------------------------------

	\begin{rSubsection}{WHISK Lab}{January 2023 - Present}{Research Assistant, Harvey Mudd College}{Claremont, CA}
		\item Worked under Professor Alexandra Schofield in the WHISK lab, in collaboration with the Data-Sitter's Club group at Stanford
		\item Analyzed the Babysitters Club book series using NLP techniques to study language found in the literature. 
		\item Employed topic models to process over 100 books, uncovering trends both across books and chapters. 
		\item This work was written up and posted online \href{https://datasittersclub.github.io/site/dsc20.html}{here} to make NLP techniques more accessible to general audiences.
	\end{rSubsection}

%------------------------------------------------

	\begin{rSubsection}{Biophotonics Lab}{January 2021 - May 2022}{Research Assistant, Harvey Mudd College}{Claremont, CA}
		\item Worked under Professor Joshua Brake in the Biophotonics Lab, Department of Engineering
		\item Investigated the effectiveness of Fourier Ptychographic Microscopy (FPM), a computational resolution-enhancement technique, in overcoming inherent physical limitations of optics systems. 
		\item Used Autodesk Inventor to print and build a UC2 Microscopy system to explore the applications of low-cost microscopy.
		\item Improved the FPM algorithm by coding and testing two alternative reconstruction techniques.
	\end{rSubsection}

\end{rSection}

%----------------------------------------------------------------------------------------
%	TECHNICAL STRENGTHS SECTION
%----------------------------------------------------------------------------------------
\begin{rSection}{Work Experience}
\begin{rSubsection}{Microsoft Corporation}{May 2023 - August 2023}{Software Engineering Intern, Azure Networking}{Redmond, WA}
		\item Designed and developed an optimized, user-centric AI-powered assistant that utilized natural language interactions to retrieve and present relevant knowledge regarding documentation from internal repositories.
        \item Established a dynamic workflow to consistently update the assistant’s knowledge base, ensuring current and precise responses in alignment with evolving resources.  
        \item Employed machine learning models and NLP techniques, including retrieval augmented generation and prompt engineering, to enhance performance and ensure the assistant’s adaptability to changes in training data. 
        \item Successfully tested and validated the tool using real queries, and subsequently deployed it for the team’s use.
	\end{rSubsection}
 \begin{rSubsection}{Pure Storage, Inc.}{May 2023 - August 2023}{Software Engineering Intern, Firmware Team}{Mountain View, CA}
		\item Implemented an API to emulate hardware functionality in a software-only environment. 
        \item Developed a multi-threaded C program within a Linux environment, leveraging inter-process communication methodologies such as POSIX message queues and shared memory. 
        \item Programmed in Python and Cython to seamlessly integrate C code with an established codebase.

	\end{rSubsection}
\end{rSection}

\begin{rSection} {Projects}
    \begin{rSubsection}{NASA Ames}{August 2023 - Present}{Harvey Mudd Clinic Team}{Claremont, CA}
        \item Collaborated with NASA engineers through Harvey Mudd’s senior capstone class
        \item Designed, documented, and prototyped from the ground up a software framework for evaluating speech-to-text systems to assess which systems are best suited for NASA’s needs. 
        \item Coded in C\# and Python to evaluate STT APIs on a variety of language processing metrics, including WER, MER, WIL, and more.
    \end{rSubsection}    
\end{rSection}

\begin{rSection} {Other Experience}
\begin{rSubsection}{Grader/Tutor}{August 2021 - Present}{Department of Computer Science}{Claremont, CA}
        \item Tutored students enrolled in Algorithms (Spring 2023 – Present), Computability and Logic (Fall 2022 – Spring 2023), Intro to Climate Change (Fall 2021), Intro to Computer Science (Fall 2021), helping explain difficult concepts.
        \item Graded students’ work, and helped professors evaluate students in the above courses. 
        \item Will be grading and tutoring for Natural Language Processing in Spring 2024. 
    \end{rSubsection} 
    
\begin{rSubsection}{Student Faculty Search Committee Member}{August 2023 - Present}{Department of Computer Science}{Claremont, CA}
        \item  Nominated by a Computer Science faculty member to take part in Harvey Mudd's Student Faculty Search Committee.
        \item Attended teaching demos, presentations, and faculty lunches to assist in evaluating CS faculty candidates. 
    \end{rSubsection} 

    \begin{rSubsection}{President}{August 2020 - Present}{South Asian Student Association}{Claremont, CA}
        \item Organized cultural events, workshops, and social networking events as president of SASA, the largest South Asian organization across the Claremont Colleges.
        \item Managed budgets, negotiated fund allocations with college administrations, and developed promotional campaigns to increase engagement among the student body. 
    \end{rSubsection} 

    
\end{rSection}
\begin{rSection}{Honors}
Harvey Mudd College Dean’s List (All Semesters)\\
National Merit Scholarship Finalist\\
Lynbrook High School Valedictorian\\
\end{rSection}
\begin{rSection}{Technical Strengths}

        
	\begin{tabular}{@{} >{\bfseries}l @{\hspace{6ex}} l @{}}
		Programming Languages & Python, Java, C, C++, C\#, R, Racket, Haskell \\
	   Languages & English, Tamil, Spanish
		
	\end{tabular}

\end{rSection}

%----------------------------------------------------------------------------------------
%	EXAMPLE SECTION
%----------------------------------------------------------------------------------------

%\begin{rSection}{Section Name}

	%Section content\ldots

%\end{rSection}

%----------------------------------------------------------------------------------------

\end{document}
